% Author: Matthew Turner

\documentclass[11pt,letterpaper]{article}
% \documentclass[11pt]{report}
% \documentclass{report}
% \documentclass{book}
\usepackage[bookmarks]{hyperref}
\usepackage{amssymb,amsmath}
% \usepackage{fullpage}
\usepackage{tabulary}
\usepackage{tabularx}
\usepackage{float}
% \usepackage[margin=1.00in]{geometry}
\usepackage[margin=0.90in]{geometry}

\usepackage{caption}
\usepackage{booktabs}
\usepackage{pslatex}
\usepackage{apacite}
\usepackage{subcaption}
\usepackage{pgfplots}
\usepackage{wrapfig}
\usepackage[english]{babel}
\usepackage{lmodern}
\usepackage{setspace}
\doublespace
% \usepackage{url}
\usepackage{bigfoot}
\usepackage[export]{adjustbox}
\setlength\intextsep{0pt}

\usepackage{graphicx}

\title{Potential errors of sign and magnitude in behavioral studies of group polarization}

\author{{Matthew A.~Turner}}

\begin{document}
\maketitle

\begin{abstract}
  ``Group polarization'' is understood as a key social structuring process
  that increases extremism and exacerbates society-level polarization. 
  In group polarization,
  isolated groups similar in their biases on some topic become more extremely biased after
  deliberation. In order to understand the causes of group polarization, it is 
  necessary to perform behavioral studies to understand what sorts of groups
  do go to extremes in which situations. However, following the published
  behavioral studies would lead one to use statistical methods that are known
  to cause errors of magnitude and sign in determining effect sizes. 
  In general, unless one has knowledge of the variances of measurements it is
  impossible to know if there has been a false inference. In many cases
  the variance of data were not reported in original studies and are now
  lost to time. Must we conclude many published results are unreliable? 
  This paper shows that we can cautiously accept the results. We support
  this claim with computational experiments that recreate three case studies
  we use to understand the effects of using metric models on ordinal data
  in the group polarization setting specifically. We close by discussing the
  effect these findings have on the state of knowledge about group polarization
  and social structuring more generally, and whether more complicated ordinal
  models are really needed over the simpler, easier-to-understand
  metric models.
\end{abstract}


\section{Introduction}

Problem:
\begin{enumerate}
  \item
    Using differences between 
    continuous model fits (e.g.\ pre- and post-discussion normal distributions) 
    to represent differences between participant responses on an ordinal 
    scale can result in false inferences about that difference
    \cite{Liddell2018}. Specifically, the error may be one of sign
    (the calculated effect size was positive but the true effect was negative)
  \item
    Many group polarization studies, including some 
    high profile ones, used metric models on ordinal data. 
    (COULD DO SOME SIMPLE METASCIENCE TO QUANTIFY HOW IMPACTFUL 
    AFFECTED STUDIES HAVE BEEN IN TERMS OF CITATIONS)
\end{enumerate}

These facts lead to the following related questions:
\begin{enumerate}
  \item 
    Must published group polarization studes be rejected due to
    false statistical inferences?
  \item 
    Assuming published studies need not be rejected, must new group 
    polarization behavioral studies use a more complex
    statistical model than, say, a t-test? 
\end{enumerate}

Now that we have introduced the general problems and questions, we will explain
in more detail just what effects are being estimated by a continuous model
of opinion change. We will do this by first introducing the group polarization
phenomenon, theoretical explanations, and supporting evidence. Then we will
brielfy introduce how our model of statistical inference for group 
polarization may result in false inferences~\cite{Liddell2018}. In our
Results we will then apply this model three case studies taken from the
group polarization literature that use metric models with ordinal data.
Through simulations using published data and some knowledge about the
group polarization study procedures we will determine whether or not
results from these individual case studies must be rejected due to 
potentially false inferences.

\subsection{Group polarization studies not affected by this problem}

The group polarization effect is considered robust. The theory of group polarization has had
broader impacts beyond social psychology, notably on political psychology
and law~\cite{Sunstein2002,Sunstein2009,Sunstein2019}. Whether or not the 
published behavioral studies we analyze here using metric models on 
ordinal data make false inferences,  there are several alternative approaches
to group polarization experimental design. I explain some here to help us
better understand what group polarization is and explaining how the 
published empirical studies under investigation here fit within the broader
scope of group polarization studies.

\begin{itemize}
  \item Gambling studies where opinions are about how much risk is 
    acceptable \cite{Blascovich1973,Blascovich1974,Blascovich1975,Blascovich1975a,Blascovich1976}
  \item Pattern-oriented approach of \citeA{Mas2013} to demonstrate their
    model predicted general patterns of observed opinion dynamics.
  \item Jury deliberations about how much money to reward plaintiffs
    \cite{Myers1976,Kaplan1977,Schkade2000}.
\end{itemize}

\subsection{Group opinion polarization experiment design}

\begin{enumerate}
  \item
    (INTRO PAR) In this paper we develop a model of group opinion shift measurement,
    detection, and quantification based on experimental design in the literature.
    To motivate our model development, here we identify a common set of
    design elements used in group opinion polarization experiments and 
    provide some examples of how existing studies vary this basic design in
    order to answer specific theoretical questions about group polarization.
    \begin{itemize}
      \item 
        There are some basic, common elements among opinion-based group polarization
        which together we call the ``basic experimental paradigm'' for reference.
        In details, opinion-based group polarization experiments can vary greatly
        as needed for testing specific hypotheses.
      \item 
        No matter the research question, the basic outcome measure is the 
        group opinion shift before and after discussion. There are various ways
        to perform the measurements of individual opinions, but all 
        group polarization studies must measure a group opinion shift. 
        In group polarization studies, several group shifts are observed, so 
        the statistical tests to quantify the magnitude and sign of choice shift
        compare the pre-discussion and post-discussion opinion distributions.
      \item
        In the Model section below we explain in more detail methodologies
        for quantifying this shift. For now we focus on reviewing existing
        approaches to experimental design for inducing group opinion shifts.
      \item
        To give an illustrative example with historical relevance, we will use
        examples of how the research question and hypotheses have guided
        experimental design. These experiments were designed
        to determine the extent to which
        group opinion polarization occurs due to either 
        ``social comparisons''~\cite{Myers1978} or ``persuasive 
        arguments''~\cite{Vinokur1978}, 
        or in some combination of the 
        two~\cite{Burnstein1973,Burnstein1977,Sieber2019}.
    \end{itemize}
  \item 
    (THE BASIC EXPERIMENTAL PARADIGM---can still use examples, but focus
    only on the high level)
  \item 
    (OVERVIEW OF VARIATIONS AND THEIR PURPOSE---eg separating social comparisons
    from persuasive arguments requires an argument-free condition; explain 
    how those work so the reader understands the scope of the model)
\end{enumerate}



\section{Model}

\begin{enumerate}
  \item 
    Model has two levels: group level and individual level. 
  \item 
    Group polarization operates at the group level. In order to
    detect group polarization, two distributions of participant opinions
    are observed: the pre-discussion and post-discussion distributions in the
    basic
  \item 
    The purpose of the current research is to understand the impact of 
    statistical model choice on the reliability of published and future
    experimental results. 
  \item 
    The distribution of reported opinions depends on
    how individual opinions were measured in the first place. 
  \item 
    This present
    study then does not develop a generative model of group polarization
    based on individual-level psychology. Instead we develop a measurement
    and distributional model of group polarization. Exactly which psychological
    processes are at work in group polarization are irrelevant when designing
    a statistical inference procedure for detecting and quantifying 
    group opinion shifts.
  \item 
    (SUBSEC) Psychology and measurement of opinions
    \begin{itemize}
      \item 
        If we use a Likert-style scale for measuring opinions we force 
        participants to report changes in their opinions on a discrete scale.
      \item Psychologically, it seems unlikely that opinions change from,
        e.g., ``Somewhat Agree'' to ``Agree'' in such discrete steps. We
        are convinced little by little (REF), though large opinion shifts
        are possible (REF).
      \item
        A further complication is that we don't know exactly how the 
        theoretical \emph{latent opinions} somehow stored in our brain connect
        to the \emph{reported opinions} given by participants in response
        to survey questions.
    \end{itemize}
  \item
    (SUBSEC) Detecting shift in group opinions
    
    \begin{itemize}
      \item 
        Common approaches in literature 
        (FOCUS ON STATISTICS USED IN CASE STUDIES TO START).
      \item 
        Common approaches are potentially problemantic~\cite{Liddell2018}.
        \begin{itemize}
          \item
            Prototypical examples adapted from Liddell and Kruschke
          \item 
            In the prototypical examples given above, certain parameters
            may be having an impact on the probability that an inference is
            false. 
            \begin{itemize}
              \item 
                One of those parameters is explicitly set to \emph{generate}
                the false inferences, the variance. 
              \item
                In order to generate false inferences in opinion shifts, 
                it was necessary that the pre-discussion variance be
                less than the post-discussion variance. 
              \item
                In group polarization experiments, the reverse is the case:
                pre-discussion opinions are initially more diverse than
                post-discussion opinions. This is due to the theoretical 
                operation of consensus formation that tends to regress opinions
                towards the mean~\cite{French1956,DeGroot1974}, even if that mean
                shifts, as it does in group polarization due to underlying
                psychological mechanisms~\cite{Sieber2019,Turner2020}.
            \end{itemize}
        \end{itemize}
    \end{itemize}

  \item
    (SUBSEC)
    Computational experiments to detect false inferences
    \begin{itemize}
      \item 
        False positives: set latent means to be exactly equal and see if
        metric model infers significant difference.
      \item 
        False negatives: set latent means to be significantly different
        and see if metric model infers no difference.
      \item 
        Errors of sign (sign inversion): set latent means to be significantly
        different in one ordering, see if model infers the inverse ordering.
        E.g., pre-discussion mean greater than
        post-discussion mean, but model infers the inverse.
      \item
        We will apply this method to three case studies in the Results below,
        from which we can begin to make broader judgments about the chance
        an average group opinion polarization study can be trusted or not.
    \end{itemize}
\end{enumerate}

\section{Resuts}

In this section we apply our model to three case studies taken from the 
literature. 

\subsection{Case study I: A tale of two cities in Colorado, USA, in 2010}

We begin by evalutating a study of attitudes about group 
polarization on political topics
in the United States. In \citeA{Schkade2010}, 
although ordinal models are used for analysis of metric data, the paper 
included information about the variance of opinions. This extra information
allows us to conclude that this study does not make false inferences due to
using a metric model on an ordinal dataset.

\begin{enumerate}
  \item Introduction of the study:
    \begin{enumerate}
      \item Specific research questions 
      \item Experimental design (participant pool, survey questions,
        methodology---measurement and analysis)
    \end{enumerate}
  \item Study data and conclusions 
  \item Representation of this experimental design with our model
  \item Model results and determination of false inference due to 
    metric model on ordinal data
\end{enumerate}

\subsection{Case study II: French attitudes about their president and Americans}

\citeA{Moscovici1969}


\subsection{Case study III: Racial attitudes in the USA in the late 1960s}

\citeA{Myers1970}

\section{Discussion}


% Make sure to RePAV your paper often. That is, check on how well you have
% stated
% \begin{enumerate}
%   \item \textbf{Re}search tradition/community you are addressing
%   \item \textbf{P}roblem you are addressing/solving
%   \item \textbf{A}pproach taken
%   \item \textbf{V}alue of the work
% \end{enumerate}

\bibliographystyle{apacite}

\setlength{\bibleftmargin}{.125in}
\setlength{\bibindent}{-\bibleftmargin}

\bibliography{/Users/mt/workspace/papers/library.bib}

\end{document}
